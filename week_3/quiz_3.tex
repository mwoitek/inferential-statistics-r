% Created 2024-06-29 Sat 20:03
% Intended LaTeX compiler: pdflatex
\documentclass[11pt]{article}
\usepackage[utf8]{inputenc}
\usepackage[T1]{fontenc}
\usepackage{graphicx}
\usepackage{longtable}
\usepackage{wrapfig}
\usepackage{rotating}
\usepackage[normalem]{ulem}
\usepackage{amsmath}
\usepackage{amssymb}
\usepackage{capt-of}
\usepackage{hyperref}
\usepackage[a4paper,left=1cm,right=1cm,top=1cm,bottom=1cm]{geometry}
\usepackage[american, english]{babel}
\usepackage{enumitem}
\usepackage{float}
\usepackage[sc]{mathpazo}
\linespread{1.05}
\renewcommand{\labelitemi}{$\rhd$}
\setlength\parindent{0pt}
\setlist[itemize]{leftmargin=*}
\setlist{nosep}
\date{}
\title{Week 3 Quiz}
\hypersetup{
 pdfauthor={Marcio Woitek},
 pdftitle={Week 3 Quiz},
 pdfkeywords={},
 pdfsubject={},
 pdfcreator={Emacs 29.4 (Org mode 9.8)}, 
 pdflang={English}}
\begin{document}

\thispagestyle{empty}
\pagestyle{empty}
\section*{Problem 1}
\label{sec:orgf6b633d}

\textbf{Answer:} A bootstrap confidence interval constructed based on a biased sample
will still yield an unbiased estimate for the population parameter of interest.
\section*{Problem 2}
\label{sec:org8bb9b52}

\textbf{Answer:}
\begin{itemize}
\item Version 1: Each observation in one data set is subtracted from the average of
the other data set's observations.
\item Version 2: The data can't be considered paired data because the days for which
we have Intel data may be different from the days for which we have Southwest
Airlines data.
\end{itemize}
\section*{Problem 3}
\label{sec:org18358c3}

\textbf{Answer:} Sampling distribution
\section*{Problem 4}
\label{sec:org34a0409}

\textbf{Answer:}
\begin{itemize}
\item Version 1: The study's authors are 95\% confident that babies born to
non-smoking mothers are on average 0.2 to 0.9 pounds heavier than babies born
to smoking mothers.
\item Version 2: Since 0 is apparently an unusual value for the statistic, then at
the 5\% significance level we would fail to reject a null hypothesis that
claims that the fathers' and mothers' average IQs are equal.
\end{itemize}
\section*{Problem 5}
\label{sec:orgb5af664}

\textbf{Answer:}
\begin{itemize}
\item Version 1: Because the standard error estimate may not be accurate.
\item Version 2: A confidence interval based on this sample is not accurate since
the sample size is small.
\end{itemize}
\section*{Problem 6}
\label{sec:orgc89b210}

\textbf{Answer:} It becomes more normal looking.
\section*{Problem 7}
\label{sec:orgaca94c1}

\textbf{Answer:} Tom has a one-sided alternative hypothesis and should do a paired
t-test.
\section*{Problem 8}
\label{sec:org60e9587}

\textbf{Answer:}
\begin{itemize}
\item Version 1: Between 0.02 and 0.05.
\item Version 2: \(T < -1.71\)
\end{itemize}

\begin{verbatim}
# Version 1
n <- 26
t <- 2.485
p_value <- 2 * pt(t, n - 1, lower.tail = F)
print(p_value)
\end{verbatim}

\phantomsection
\label{}
\begin{verbatim}
0.0200048010978179
\end{verbatim}


\begin{verbatim}
# Version 2
print_p_value <- function(t, n) {
  p_value <- pt(t, n - 1)
  cat("T =", t, "--- p-value =", p_value, "\n")
}

n <- 25
ts <- c(-1.71, -1.32, 1.32, 1.71, 1.96)
for (t in ts) {
  print_p_value(t, n)
}
\end{verbatim}

\phantomsection
\label{}
\begin{verbatim}
T = -1.71 --- p-value = 0.05008269 
T = -1.32 --- p-value = 0.09964372 
T = 1.32 --- p-value = 0.9003563 
T = 1.71 --- p-value = 0.9499173 
T = 1.96 --- p-value = 0.969147 
\end{verbatim}
\section*{Problem 9}
\label{sec:org0811dfa}

\textbf{Answer:} F-test (ANOVA)
\section*{Problem 10}
\label{sec:org7fa9f83}

\textbf{Answer:}
\begin{itemize}
\item Version 1: There should be at least 10 successes and 10 failures.
\item Version 2: Side-by-side box plots showing roughly equally sized boxes for each
group.
\end{itemize}
\section*{Problem 11}
\label{sec:org51b070f}

\textbf{Answer:} 1.87

\begin{verbatim}
p <- 1 - 0.0767
df1 <- 7
df2 <- 189
f_stat <- round(qf(p, df1, df2), digits = 2)
print(f_stat)
\end{verbatim}

\phantomsection
\label{}
\begin{verbatim}
1.87
\end{verbatim}
\section*{Problem 12}
\label{sec:org592940f}

\textbf{Answer:}
\begin{itemize}
\item Version 1: At least two group means are significantly different from each
other.
\item Version 2:
\begin{align*}
  H_0&:\mu_1=\mu_2=\mu_3=\mu_4=\mu_5\\
  H_A&:\text{at least one }\mu_i\text{ is different}
\end{align*}
\end{itemize}
\section*{Problem 13}
\label{sec:orgf2db6fb}

\textbf{Answer:}
\begin{itemize}
\item Version 1: Decrease
\item Version 2: 1.023
\end{itemize}

\begin{verbatim}
# Version 2
n1 <- 18
s1 <- 3.4

n2 <- 18
s2 <- 2.7

sp <- sqrt(((n1 - 1) * s1^2 + (n2 - 1) * s2^2) / (n1 + n2 - 2))
std_err <- sp * sqrt(1 / n1 + 1 / n2)
signif(std_err, digits = 4)
\end{verbatim}

\phantomsection
\label{}
\begin{verbatim}
[1] 1.023
\end{verbatim}
\section*{Problem 14}
\label{sec:org9e09806}

\textbf{Answer:}
\begin{itemize}
\item Version 1: \(\alpha^{\prime}=0.005\)
\item Version 2: \(K=10\)\\
\end{itemize}

The adjusted significance level is given by
\begin{equation}
\alpha^{\prime}=\frac{\alpha}{K},
\end{equation}
where \(K\) can be obtained from the number of groups \(k=5\) as follows:
\begin{equation}
K=\frac{k(k-1)}{2}=\frac{5(5-1)}{2}=10.
\end{equation}
Using this result and \(\alpha=0.05\), we get
\begin{equation}
\alpha^{\prime}=\frac{0.05}{10}=0.005.
\end{equation}
\end{document}
